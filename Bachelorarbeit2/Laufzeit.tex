Auf einen Apple MacBook M1 ergeben sich folgende Zeiten: Bei einer Eingabe von $10^7$ Elementen hat die Implementierung mit einer Liste nach einer halben Stunde noch kein Ergebniss geliefert, bei der Implementierung mit FT ergab sich eine Zeit von ca. 7 Sekunden. Bei kleineren Längen gab der DP Algorithmus und die Abwandlung aus Kapitel 4 keine nennenswerten Unterschiede, welche in einer langsamen Datenstruktur wie der Liste eigentlich zu erwarten wären, deswegen schließe ich hier, dass beide Algorithmen gleichwertig benutzt werden können. Der Unterschied zwischen Liste und FT ist aber nicht mehr im Promille Bereich, somit ist eine Implementierung mit einem FT immer einer mit verketteter Liste vorzuziehen. Der Grund für diese Laufzeitunterschiede ist die effektive Benutzung der binären Operationen im FT und die langsame Implementierung von $prev$ in einer Liste. Weiter ist die Verwendung von expliziten Referenzen in der Liste ein großer Faktor, der im FT wegfällt. 